The purpose of this thesis is to test a new, weakly context-dependent multi-armed bandit algorithm that not only balances exploration and exploitation, but also selects never-before-seen arms for each user-context. By assuming that the reward from each arm is IID, this new algorithm leverages all historical data (regardless of user-context) but still allows a certain level of user-specific recommendations. This, in turn, allows the algorithm to scale well over a large user base.

In order to test this new algorithm (which I call UCB1-AKSB), I implemented it in Tigers Anonymous (TA) and optimized the algorithm to select conversation starters in order to maximize conversation de-anonymizations. After collecting and analyzing conversation and user data, we can draw the following conclusions about the UCB1-AKSB algorithm.

First, the data gathered suggests that TA violated some of the key assumptions made when applying the UCB1-AKSB algorithm. For example, the two different regimes in the cumulative regret analysis in \autoref{sec:RegretAnalysis} suggest that the distribution of payouts for each bandit arm were not IID as assumed, but rather weakly Markovian. However, for the second regime, the algorithm behaved as expected and produced logarithmic cumulative regret, in line with the canonical UCB1 algorithm. Moreover, the process of TA conversation de-anonymization may have been more context-dependent than UCB1-AKSB accounted for.

Second, it seems that UCB1-AKSB did not improve the overall conversation de-anonymization rate (although it improved the proportion of de-anonymizations given the option to de-anonymize). This is probably because conversation de-anonymization was not only dependent on the conversation starter and the binary variable of de-anonymization was not fine-grained enough to measure incremental improvements in the de-anonymization rate. However, UCB1-AKSB resulted in noticeable improvement in conversation quality over time on the basis of other metrics of conversation quality. In addition, UCB1-AKSB resulted in consistent improvement in per-user performance in all three of the conversation quality metrics used in this paper: conversation de-anonymization, total conversation length and participation rate.

Finally, the data suggests that, over time, the TA user base became stratified into two categories, power-users and amateur-users, with little middle ground. This resulted in increasing volatility in per-user conversation quality metrics as explained in \autoref{sec:IndividualUserAnalysis}. It also resulted in user saturation, which decreased the overall TA user experience and likely contributed to the decline in the usage of the site.

In all, it seems that the UCB1-AKSB algorithm performed quite well given the violation of some of its fundamental assumptions, its calibration on an imperfect metric and the stratification/saturation of its user base. However, this thesis was only a single empirical test of this new algorithm, which leaves substantial room for further testing. For future research in this area, this thesis can make some important recommendations.

First, any future research would benefit from a larger user base to avoid the problems of user saturation and stratification mentioned in \autoref{sec:IndividualUserAnalysis}. This would make the formation of a `casual-user' class more likely, which would bridge the gap between amateur and power-users, thus improving the user experience and making the chatroom gain more traction with the user base.

Second, the algorithm's performance could be improved by using a better measure of conversation quality than a simple binary variable (i.e. the likelihood of de-anonymization). For example, one could create a hybrid metric by assigning a more finely tuned score based on multiple metrics (i.e. 0 for immediate disconnect, 0.3 for a short conversation, 0.7 for a long conversation, and 1 for a de-anonymization). Another option would be to remove the binary variable of de-anonymization altogether, and instead use another metric, such as the participation rate or average conversation length. 

Finally, the algorithm itself could be modified to work more effectively with a static set of arms or a dynamic set of arms with a slow turnover rate. The current version of the algorithm outlined in \autoref{ch:Methods} simply serves a random arm if both user pairs have already seen the conversation starter. This could be improved by implementing a sliding time window for context dependency - that is, only use the conversation starters that users $u$ and $v$ have not seen within the last week or month. This could also be implemented as a sliding number of plays (i.e. only pick from arms that users $u$ and $v$ haven't seen within the last 10 plays).

Not only is there room for future research on the UCB1-AKSB, but the algorithm has broad applications beyond simply selecting quirky conversation starters for an anonymous chatroom. For starters, it is very fast and computationally easy to implement, as opposed to more intensive contextual bandit algorithms such as LinUCB in \citet{chu10}. In addition, UCB1-AKSB (or at least some of its principles) can be applied to nearly every bandit recommendation algorithm. In stochastic bandit recommendation systems that would benefit from a degree of context-dependency, such as making novel recommendations to a homogenous user base, the UCB1-AKSB would provide just enough context dependency while still taking advantage of the assumption that user responses are IID. It is easy to see this algorithm being applied to recommend news articles or movies of a certain genre, where novelty is important and users are relatively homogenous. In other bandit algorithms that are more heavily context dependent, the arm-filtering mechanism used by UCB1-AKSB could be added on top of the existing bandit mechanism to ensure that users do not see an arm more than once.

In conclusion, the UCB1-AKSB algorithm fills a gap in the literature between stochastic multi-armed bandits and context-dependent bandits. It provides a solution to a real-world problem in Tigers Anonymous and could be applied in a variety of other contexts. By providing a weakly context-dependent multi-armed bandit algorithm and showing its potential for performance in real-world applications, I hope that this thesis will help spur future research into this new class of algorithms.
