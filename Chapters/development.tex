%%% This is the development.tex

This Chapter has purposefully a very long title to illustrate how
\LaTeXe{} handles such long names. Now it is a good time to look on
the table of contents and see how this Chapter and also
Chapter~\ref{ch:intro} are listed. Notice that the number 1 in the
phrase ``and also Chapter 1'' is automatically generated by using
the command \verb|\ref{ch:intro}|, because we labeled
Chapter~\ref{ch:intro} `\verb|ch:intro|' by using the command
\verb|\label{ch:intro}| after the \verb|\chapter{Introduction}|
declaration.

\section{Initial Setup}
A few things here to start. And here as well, since we need to fill
in at least a line. Almost\ldots, there we go!

Notice that I entered ellipses after the word `Almost' above with
the command \verb|\ldots|, and not by simply typing three dots.
Compare the result here: \ldots vs. ...\footnote{Word does that too,
but lousily!}

Maybe some more words in a new paragraph. And more, and more, and
more. Furthermore, additionally, in addition, and so on. Notice here
that the word `Furthermore' was broken into `Fur-' and `thermore' in
order to fit the line---Word, as stupid as it is, would simply place
the entire word on the next line, thus increasing the distance
between words on the first line to fill up the entire line.

\subsection{Additional Structure: The Use of Subsections}
\label{sec:structure} We are in a subsection now (two levels down
from a Chapter). When we refer to $X.Y.Z$, we mean Chapter $X$,
Section $Y$, and subsection $Z$. We declare a Chapter by the command
\verb|\chapter{|\emph{title of Chapter}\verb|}|, a Section by
\verb|\section{|\emph{title of Section}\verb|}|, and so on.

The nice thing about \LaTeX{} is that it takes care of the chapter,
section, and subsection numbering automatically. If I were to add
another subsection before this one the subsection number would
change (increment by one). This section is \ref{sec:structure} and I
referred to it using the command \verb|\ref{|\emph{label of this
section}\verb|}|. I inserted a label right after the
\verb|\subsection| declaration by typing \verb|\label{|\emph{label
of this section}\verb|}|.

\subsubsection{A subsubsection}\label{subsub} Just for fun! Notice
that no number is alloted for such a low level environment but it
sometimes useful.

\subsection{Another Subsection}
And so on\ldots.

\section{Mathematical Symbols}
Let $X=\{X_n, n\in \N\}$ be a Markov chain with state space
$\mathcal{D}$. Throughout this thesis, we use the notation
\begin{equation}
p_{ij} := \P\{X_{n+1} = j \mid X_n = i\}, \quad i,j \in \mathcal{D}
\label{pij}
\end{equation}
for the transition probabilities of the Markov chain $X$.
Furthermore, we denote by $P$ the transition matrix, $P =
[p_{ij}]_{i,j\in\mathcal{D}}$.

When we wrote \eqref{pij} we implicitly assumed that the Markov
chain $X$ is time-homogeneous.

Let us also define $Y$,
\begin{equation*}
Y = (Y_n)_{n = 0,1,2,\ldots}
\end{equation*}
to be another process. Notice that the second equation does not take
a number on the right---this is the use of \verb|\begin{equation*}|
environment.

Notice that the all the math characters, $X$, $\mathcal{D}$, and
others such as $\alpha, \beta, \gamma$ are part of the text in
\LaTeX{}. On the contrary, Word includes such characters as foreign
objects (usually images), which increases the size of the document
file, sometimes makes them disappear, but most importantly are not
as aesthetically pleasing as the resulting characters here.

\section{Citing and Bibliography}
When working with large documents you need an easy way to cite your
references without having to go back to your list all the time to
remember the names of the authors and the year of publication. Even
more importantly, you need to have all your references listed in the
end of the document in alphabetical order. Of course, they all need
to be syntactically the same so that alone makes the manual entry of
references a big pain. Thankfully, \LaTeX{} takes care of that in a
very easy and elegant way, using \BibTeX.

I cite here a few books, papers, and technical reports, and please
go to page \pageref{bib} to see the resulting bibliography.

According to the books by \cite{C75}, \cite{BR02}, and \cite{MR97}
and the articles by \cite{DG01}, \cite{BBM05}, and \cite{CFPS04} we
conclude absolutely nothing. However, in his report, \cite{A04}
claims that otherwise. All these citations were entered by \verb|\cite{|\emph{citation label}\verb|}|.
 
Notice the different citation style that follows: it is parenthetical, and observe that only one pair of parentheses is required \cite[see Theorem 5.2][pg. 32]{AMM05}. This citation is entered by typing \verb|\cite[see Theorem 5.2][pg. 32]{AMM05}| in the \verb|.tex| file. (Here, the citation label corresponding to \cite{AMM05} is obviously \verb|AMM05|.)


The citations are included in the file \verb|refs.bib| under the
folder \verb|Bibliography|. You can modify it and make your own
references. I highly recommend using \emph{JabRef} for managing your bibliography entries, because it makes it a piece of cake to do a lot of \emph{dirty} work. \emph{JabRef} is free and it works as a Java Application.

Also notice that \LaTeX{}, by default, includes in the Bibliography section only the references you actually cited throughout the text. If you want a source to appear in the Bibliography section without actually citing it anywhere in your text use the command \verb|\nocite{|\emph{citation label}\verb|}|. For example here I type \verb|\nocite{B95}| \nocite{B95} and you see no citation appear---however look at the fourth entry of the Bibliography. That cited book does not appear anywhere in this thesis, other than the Bibliography.

\section{Referencing Figures and Tables}
The very informative Figure~\ref{fig:dens} is on page~\pageref{fig:dens}. Both of these numbers were automatically generated---which is great when you add a new figure before the one you just inserted, because the numbering changes automatically for you. Use \verb|\ref{fig:dens}| for the figure number and \verb|\pageref{fig:dens}| for the page number where the figure is located. Here, \verb|fig:dens| was the label of the figure (see actual \verb|.tex| file for more information). Remember that \LaTeX{} does not work like Word---the figures and tables are \textbf{not} always placed exactly where you want them, so avoid writing ``according to the figure below\ldots,'' and prefer writing ``according to Figure~[\emph{figure number}]\ldots,'' instead. The same things go unchanged for tables. Notice that when I talk about figures and tables in general, I do not need to capitalize them, however if I talk specifically about Figure~\ref{fig:dens} and Table~\ref{tab:cdo}, I'd better respect them and capitalize the `f' and the `t.'

Since we're at it, notice that the quotes `, ', ``, '' are not inserted like in Word. For ` you need to use the \verb|`| key that is located above the \verb|Tab| button. For ' you just press the \verb|'| key, exactly to the left of the \verb|Enter| key. For double quotes just double the appropriate single quotes without leaving any space.
