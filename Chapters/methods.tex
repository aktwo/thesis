\section{UCB1-AKSB Algorithm}

The multi-armed bandit algorithm used by Tigers Anonymous, named UCB1-AKSB, is a novel variant of the well-known UCB1 algorithm \citep{auer02}. Before explaining the algorithm, it will be useful to introduce notation. Let the users be represented as the set $U$ and the bandit arms as the set $X$. Let the set of arms that have already been played for user $u\in{U}$ be represented by the set $X^u \subset{X}$. The goal of the UCB1-AKSB algorithm is to pick some arm $x\in{X}$ given the pair of users $u,v \in{U}$. In this specific application, the goal is to pick the optimal conversation starter $x\in{X}$.

The UCB1-AKSB algorithm proceeds as follows: For each pair of users $u,v\in{U}$, we pick the conversation starter $x$ such that

\begin{equation}
\label{eq:UCBMain}
x = \underset{x \in{\{X^u \cup X^v\}}^{\mathsf{c}}}{\arg\max{}} f(x)
\end{equation}
where

\begin{equation}
\label{eq:UCBMetric}
   f(x) = \left\{
     \begin{array}{lr}
       \bar{x}+ \sqrt{\frac{2\ln{n}}{n_x}} & : n_x > 0\\
       \infty & : n_x = 0
     \end{array}
   \right.
\end{equation}

In \autoref{eq:UCBMetric}, $n_x$ is the number of times that conversation starter $x$ has been played and $n$ is the total number of conversation starters that have been shown. Note that ties are broken arbitrarily. Additionally, in the event that $\{X^u \cup X^v\} \in \emptyset$, the algorithm simply selects a random arm.

\section{Tigers Anonymous Data Collection Methods}

The complete data-collection method used for this thesis is outlined below: 

\begin{enumerate}
\item Two users visit www.tigersanonymous.com/chat from a Princeton IP address.
\item The users are directed to the chat server and are matched on a first-come, first-served basis.
\item A conversation starter is selected based on the UCB1-AKSB algorithm described above.
\item After either of the users disconnects, a 10-tuple representing the chat session is logged in a database (see \autoref{sec:TADataFormat} for more details).
\end{enumerate}

\section{Tigers Anonymous Data Format}
\label{sec:TADataFormat}

The data that will be collected can be represented by the vector of 10-tuples of the form: $$(x_i, y_i, t^0_i, t^1_i, q_i, b_i, c^1_i, c^2_i, m^1_i, m^2_i)$$ where $x_i$ and $y_i$ represent the pseudonymous user ids of the two participants in the chat, $t^0_i$ and $t^1_i$ represent the start and end times of the conversation, $q_i$ represents the conversation starter, $b_i \in {(0, 1)}$ represents whether the drop-down menu was displayed (i.e. both chat participants exchanged more than a predefined number of messages), $c^1_i, c^2_i \in{(0, 1)}$ represent whether users $x_i$ and $y_i$ opted to de-anonymize the conversation respectively and $m^1_i, m^2_i \in{(0,1)}$ represent the number of messages that user $x_i$ and $y_i$ sent respectively. The subscript $i$ is unique for each conversation. 

A sample of this data is shown below: 

\begin{lstlisting}[language=java]
[
  { 
    "userID1" : "a8262bb13e641e2bf5dcb3985b2061be",
    "userID2" : "9a675a6f581fd1dfa0b982826e75b4f5",
    "question" : "Do you believe in soul mates?",
    "startTime" : 1390873219878,
    "endTime" : 1390873263469,
    "buttonDisplayed" : false,
    "user1Clicked" : false,
    "user2Clicked" : false,
    "user1MessagesSent" : 1,
    "user2MessagesSent" : 2,
    "_id" : "52e70aafc43b6d020079e52e",
    "__v" : 0
  },
  {
    "userID1" : "370f85e443ad3ee24a879b1ce5a2b54b",
    "userID2" : "6e0fe76fca80cf2920bd5fc7717cf6dd",
    "question" : "What's one thing that you learned this week?",
    "startTime" : 1390881063228,
    "endTime" : 1390882681992,
    "buttonDisplayed" : true,
    "user1Clicked" : true,
    "user2Clicked" : true,
    "user1MessagesSent" : 33,
    "user2MessagesSent" : 37,
    "_id" : "52e72f79c43b6d020079e531",
    "__v" : 0 
  }, 
  ...
]
\end{lstlisting}

\section{TA UCB1-AKSB Implementation}

This is the code on the Tigers Anonymous chat server that implements the UCB1-AKSB algorithm.

\lstinputlisting{./Code/ucb.js}
